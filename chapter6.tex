\chapter{Summary of appended papers}
\label{ch:Summary of appended papers}
\section{Paper A}
\textit{A tablet computer application for conceptual design} \\
D. Åkesson, Lund University \\
J. Lindemann, Lund University \\
Accepted for publication in Engineering and Computational Mechanics, 2015

\section*{Summary}

A tablet computer application (named Sketch a Frame) has been developed for conceptual design. The application uses a direct manipulation cycle, where the result is computed and visualized in real time once the structural model is stable. This moves away from the step-by-step workflow of conventional structural analysis software. If the model is not stable and a force is applied, the first modal shape is visualized in the form of an animation.

The application presents various computational results, such as the normal force, moment envelope, stress, and normalized redundancy. No numerical values are presented to the user; this is a design decision to encourage the user to focus on the general structural behavior rather than the exact numerical results. 

The application has a very direct manipulation user interface that has not previously been used for this type of application.

\subsection*{Contribution}
As first author, I performed all of the new implementations involved in this project. I also wrote the paper. Jonas Lindemann developed the initial idea, and supervised the work.

\newpage
\section{Paper B}
\textit{Using 3D direct manipulation for real-time structural design exploration} \\
D. Åkesson, Lund University \\
C. Mueller, Massachusetts Institute of Technology \\
Submitted

\subsection*{Summary}

A proof-of-concept for a conceptual design application has been developed, with an unprecedented very direct manipulation user interface for 3D. A pre-existing application named ObjectiveFrame \cite{lindemann2001objectiveframe} has been combined with a 3D input device named the Leap Motion controller, allowing the user to directly interact with a structural model using hand gestures.

Three different cases were implemented:

\begin{itemize} 
\item \textit{Structural feedback} – The user can apply a force to a structure and manipulate the model by interacting with their hands. This creates a metaphor for the user identifying with how the structure feels.
\item \textit{Performance feedback} – The user can move nodes by interacting with the model using their hands. A performance index is presented to the user to provide feedback on how geometric manipulations affect the structural performance.
\item \textit{Dynamic relaxation} – The dynamic relaxation method is used to model the manipulation of a gravity load and a point load. This creates an interactive case in which the structure constantly converges to the static equilibrium and the results are visualized using an animation .
\end{itemize} 

\subsection*{Contribution}
As first author, I performed all of the new implementations involved in this project. I also wrote the paper. Caitlin Mueller supervised the work and shared ideas on the development of the application. She also assisted in communicating the novelty of the application in the paper.

