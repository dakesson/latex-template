\null\vfill

{\Huge \textbf{Populärvetenskaplig sammanfattning}} \\ \\

Det första skedet när ett byggnadsverk ska byggas är det konceptuella designskedet, det är här de första besluten om utformningen av byggnadsverket görs. I det traditionella arbetsflödet så utformas byggnadsverket först i ett ritprogram, sedan används ett annat datorverktyg för att verifiera att att det utformade byggnadsverket kan hantera de krafter som uppstår, t.ex. vind, egentyngd osv. Detta arbete är ett försök till att förbättra detta arbetsflöde genom att skapa nya verktyg som redan i ett konceptuellt designskede ger användaren en återkoppling till hur strukturen kan hantera de krafter som kommer att uppstå så att förbättringar på utformningen kan göras i ett tidigare skede.

För att skapa denna typen av verktyg så ställs höga krav på användargränssnitt, de behöver vara interaktiva och enkla att arbete med för att kunna följa designerns iterativa arbetsflöde. Direkt manipulation är en typ av användargränssnitt där användaren direkt kan manipulera objekt på skärmen. Denna typen av användargränssnitt skapar intuitiva gränssnitt som är enkla att förstå och som uppmanar användaren att experimentera och utforska möjligheter.  Ny teknik skapar nya möjligheter att skapa nya designverktyg för konceptuell design som använder en sådan, väldigt direkt, gränssnittstil. Dessa nya möjligheter utforskas vidare i denna avhandling genom att utveckla två olika designverktyg.

Det första av dessa verktyg tillåter väldigt direkt manipulation för ett två-dimensionellt (2D) användargränssnitt genom att utnyttja moderna pekskärmar. I det utvecklade verktyget kan användaren enkelt mata in och manipulera modeller genom att använda pekskärmen. Användaren presenteras med resultat från strukturberäkningar i real-tid, vilket tillåter användaren att experimentera och utforska nya former. Det andra verktyget som har utvecklats vidareutvecklar dessa koncept och idéer till tre-dimensioner (3D) genom att använda en ny sorts 3D input enhet, som heter Leap Motion.

Resultatet från detta arbetet är ett förslag på hur verktyg för konceptuell design kan förbättras genom att interagera strukturella aspekter, detta sker med hjälp av väldigt direkta användargränssnitt.


\vfill\vfill\vfill\vfill\null