% Must come in the beginning. Changes the spacing in the table of contents to look more pleasing
\usepackage{tocloft}
\setlength{\cftbeforepartskip}{5.0mm}
\setlength{\cftbeforechapskip}{2.0mm}
\setlength{\cftbeforesecskip}{0.0mm}

\usepackage{graphicx} % For \includegraphics
\usepackage{amsfonts,amsmath,amssymb} % Math symbols   
\usepackage{pdfpages} % To include the PDFs of the papers
\usepackage{titlesec}
\usepackage{emptypage}
\usepackage[hidelinks]{hyperref}
\usepackage{cite}
% Nice looking tables with correct spacings 
\usepackage{booktabs}
% Tables spanning more than one page
\usepackage{longtable}
% For tables spanning the full text width
\usepackage{tabularx}

% For a nomenclature section
\usepackage{nomencl}

\usepackage{type1cm}
\usepackage{lettrine}

% Hyphenation, support for different languages, last one is default
% Make sure to install the hyphenation packages for all languages you need

\usepackage[T1]{fontenc}
\usepackage[utf8]{inputenc}
\usepackage[swedish,british]{babel} 

\usepackage{pgfplots} 
\pgfplotsset{compat=1.8} 
%\usepgfplotslibrary{external} % Generare figurerna som pdf-er också
\usetikzlibrary{intersections}
\pgfplotsset{every tick/.style={black,}} % Gör alla små ``ticks`` svarta

% These are colors used by the LU
\definecolor{bronze}{rgb}{0.6117647059,0.3803921569,0.0784313725}
\definecolor{LUblue}{rgb}{0,0,0.5019607843}
\definecolor{LUpink}{rgb}{0.9137254902,0.768627451,0.7803921569}
\definecolor{LUcyan}{rgb}{0.7254901961,0.8274509804,0.862745098}
\definecolor{LUgreen}{rgb}{0.6784313725,0.7921568627,0.7215686275}
\definecolor{LUtan}{rgb}{0.8392156863,0.8235294118,0.768627451}
\definecolor{LUgrey}{rgb}{0.74190196078,0.7215686275,0.6862745098}
\definecolor{LUlightblack}{rgb}{0.3019607843,0.2980392157,0.2666666667}
 

% This probably did something really clever 
% -------------------------------------------------------------------
\makeatletter
\renewcommand\part{%
  \if@openright
    \cleardoublepage
  \else
    \clearpage
  \fi
  \thispagestyle{empty}%   % Original »plain« replaced by »emptyx
  \if@twocolumn
    \onecolumn
    \@tempswatrue
  \else
    \@tempswafalse
  \fi
  \null\vfil
  \secdef\@part\@spart}
\makeatother
% --------------------------------------------------------------------
 
%
 
\usepackage{ifxetex}
\ifxetex
  \usepackage[garamondx]{newtxmath}
  % If you think the integral sign is hideous in newtxmath, then use the following line instead
  %\usepackage[garamondx,cmintegrals]{newtxmath}
  \usepackage{bm}
  \usepackage[no-math]{fontspec}  
  \setromanfont[Ligatures=TeX]{Adobe Garamond Pro}
  \setsansfont{FrutigerLTStd-Bold}
  \usepackage{mathspec}
  \setmathfont(Digits){Adobe Garamond Pro}
\else
  \usepackage{bm}
\fi
\usepackage{fonttable}


% Additional font sizes \HUGE and \ssmall
\usepackage{moresize}

% Figure caption labels in bold (i.e. "Figure 1" in bold), hanging label
\usepackage{subfig}
\captionsetup{margin=0em,font={normal,rm},labelfont={bf},format=hang} 

% Figures centred, captions on top for tables
\usepackage{floatrow}
\floatsetup[table]{position=top}

\titleformat{\part}[display]
{\bfseries\HUGE\filcenter} % changes size and fonttype
{Part \thepart}{.5em}{\bfseries}
 
\renewcommand{\chaptername}{} % removes the word 'Chapter' in the titles
 
 

\titleformat{\chapter}[block] % change definition for chapter
{\bfseries\Huge\rmfamily}    % changes size and fonttype
{\thechapter}{.5em}{\bfseries}

\titleformat{\section}[block] %change definition for section
{\bfseries\Large\sffamily\MakeUppercase}    % changes size and fonttype
{\thesection}{.5em}{\bfseries\MakeUppercase}

\titleformat{\subsection}[block] % change definition for subsection
{\bfseries\large\sffamily}       % changes size and fonttype
{\thesubsection}{.5em}{\bfseries}
 
\titleformat{\subsubsection}[block]  % change definition for subsection
{\bfseries\sffamily}                 % changes size and fonttype
{\thesubsubsection}{.5em}{\bfseries}
 


\renewcommand{\bibname}{References}

\renewcommand\floatpagefraction{0.8}
\renewcommand\topfraction{0.8}
\renewcommand\bottomfraction{0.8}
\renewcommand\textfraction{0.1}


% Non-indented paragraphs with a small vertical space in-between
\parindent 0in
\parskip 3mm

% not every page needs to go to the same bottom line. Allows nicer page breaks.
\raggedbottom

\usepackage[paperwidth=178mm, paperheight=252mm, top=2.4cm, bottom=2cm, left=2cm, right=2cm]{geometry}

\usepackage{fancyhdr}
\setlength{\headheight}{15pt}

\pagestyle{fancy}
\renewcommand{\chaptermark}[1]{ \markboth{\thechapter\ #1}{} }
\renewcommand{\sectionmark}[1]{ \markright{\thesection\ #1}{} }

\fancyhf{}
\fancyhead[LE,RO]{\thepage}
\fancyhead[RE]{\textit{ \nouppercase{\leftmark}} }
\fancyhead[LO]{\textit{ \nouppercase{\rightmark}} }



% avoid orphan/widow lines. Lower this number if necessary to get a good layout. 
\widowpenalty=1000
\clubpenalty=1000
\usepackage{siunitx}

% Remove page numbers on ''chapter`` pages
\makeatletter
\renewcommand\ps@plain{\let\@mkboth\@gobbletwo
     \let\@oddhead\@empty
     \def\@oddfoot{\reset@font\hfil}
     \let\@evenhead\@empty\let\@evenfoot\@oddfoot}
\makeatother

